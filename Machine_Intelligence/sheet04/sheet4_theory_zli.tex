\documentclass[10pt]{article}

\usepackage[utf8]{inputenc}
\usepackage[margin=1in]{geometry}
\usepackage{natbib}
\usepackage{graphicx}
\usepackage{amsmath}
\usepackage{amssymb}
\usepackage{mathtools}

\title{Machine Intelligence 1 - Exercise 4: Multilayer Perceptrons and Backpropagation Algorithm}
\author{
  Liu, Zhiwei\\
  \texttt{387571}
  \and
  Moon, Chulhyun\\
  \texttt{392865}
  \and
  Wenzel, Daniel\\
  \texttt{365107}
  \and
  Ozmen, Cengizhan\\
  \texttt{388011}
  \and
  Pipo, Aiko\\
  \texttt{390011}
}
\date{\today}



\begin{document}

\maketitle

\hrule

\section*{H4.1: Line search (4 points)}

\subsection*{(a)}

The multi-dimensional Taylor approximation of a fucntion $f$ around a given point $a$ can be expressed using multi-index notation:
$$T_nf(x;a) = \sum_{|\alpha|=0}^n \frac{(x-a)^\alpha}{\alpha!}D^\alpha f(a)$$
Applying this to our case and setting $\underline{w}_{t+1} = \underline{w}_{t} - \eta_t\underline{d}_t$ yields:
\begin{equation*}
\begin{aligned}
T_2E^T(\underline{w}_{t+1};\underline{w}_{t}) & = E^T(\underline{w}_{t}) + \sum_{i=1}^d ({\underline{w}_{t+1}}_i - {\underline{w}_{t}}_i)\frac{\delta E^T}{\delta \underline{w}_i} + \sum_{i=1}^d\sum_{j=1}^d ({\underline{w}_{t+1}}_i - {\underline{w}_{t}}_i)({\underline{w}_{t+1}}_j - {\underline{w}_{t}}_j) \frac{\delta E^T}{\delta\underline{w}_{i}\delta\underline{w}_{j}} (\underline{w}_{t})\\
& = E^T(\underline{w}_{t}) + \sum_{i=1}^d (-\eta_t{\underline{d}_t}_i)\frac{\delta E^T}{\delta \underline{w}_i} + \sum_{i=1}^d\sum_{j=1}^d (-\eta_t{\underline{d}_t}_i)(-\eta_t{\underline{d}_t}_j) \frac{\delta E^T}{\delta\underline{w}_{i}\delta\underline{w}_{j}} (\underline{w}_{t})\\
& = E^T(\underline{w}_{t}) - \eta_t \underline{d}_t^T(\nabla E^T(\underline{w}_t)) + \eta_t^2 \underline{d}_t^T \underline{H}(\underline{w}_t)\underline{d}_t\\
\end{aligned}
\end{equation*}

\subsection*{(b)}
Using the inequation and $E^T(\underline{w}_{t+1}) \approx T_2E^T(\underline{w}_{t+1}; \underline{w}_t)$, we find:
\begin{equation*}
\begin{aligned}
E^T(\underline{w}_{t+1}) & \leq E^T(\underline{w}_{t})\\
E^T(\underline{w}_{t}) - \eta_t \underline{d}_t^T(\nabla E^T(\underline{w}_t)) + \eta_t^2 \underline{d}_t^T \underline{H}(\underline{w}_t)\underline{d}_t & \leq E^T(\underline{w}_{t})\\
- \eta_t \underline{d}_t^T(\nabla E^T(\underline{w}_t)) + \eta_t^2 \underline{d}_t^T \underline{H}(\underline{w}_t)\underline{d}_t & \leq 0\\
\eta_t^2 \underline{d}_t^T \underline{H}(\underline{w}_t)\underline{d}_t & \leq \eta_t \underline{d}_t^T(\nabla E^T(\underline{w}_t))\\
\end{aligned}
\end{equation*}
For $\underline{d}_t^T \underline{H}(\underline{w}_t)\underline{d}_t > 0$ and $\underline{d}_t^T(\nabla E^T(\underline{w}_t)) < 0$, we can only choose $\eta_t = 0$. In all other cases we can simplify the inequation:
$$\eta_t \underline{d}_t^T \underline{H}(\underline{w}_t)\underline{d}_t \leq \underline{d}_t^T(\nabla E^T(\underline{w}_t))$$
We obtain the following cases (for simplicity $H = \underline{d}_t^T \underline{H}(\underline{w}_t)\underline{d}_t$ and $E = \underline{d}_t^T(\nabla E^T(\underline{w}_t))$):
\begin{itemize}
\item $\eta_t \leq \frac{E}{H}$, if $H,E < 0$ or $H,E > 0$\\
\item $\eta_t> 0$ arbitrary, if $H < 0, E > 0$
\end{itemize}

\subsection*{(c)}
With this cost function we have
$$T_2E^T(\underline{w}_{t+1};\underline{w}_{t}) = $$

\end{document}
